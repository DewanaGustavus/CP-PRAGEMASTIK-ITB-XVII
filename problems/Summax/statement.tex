\documentclass{article}

\usepackage{geometry}
\usepackage{amsmath}
\usepackage{graphicx, eso-pic}
\usepackage{listings}
\usepackage{hyperref}
\usepackage{multicol}
\usepackage{fancyhdr}
\pagestyle{fancy}
\fancyhf{}
\hypersetup{ colorlinks=true, linkcolor=black, filecolor=magenta, urlcolor=cyan}
\geometry{ a4paper, total={170mm,257mm}, top=10mm, right=20mm, bottom=20mm, left=20mm}
\setlength{\parindent}{0pt}
\setlength{\parskip}{0.3em}
\renewcommand{\headrulewidth}{0pt}

\rfoot{\thepage}
\fancyhf{} % sets both header and footer to nothing
\renewcommand{\headrulewidth}{0pt}
\lfoot{\textbf{CP PRAGEMASTIK ITB XVII}}
\pagenumbering{gobble}

\fancyfoot[CE,CO]{\thepage}
\lstset{
    basicstyle=\ttfamily\small,
    columns=fixed,
    extendedchars=true,
    breaklines=true,
    tabsize=2,
    prebreak=\raisebox{0ex}[0ex][0ex]{\ensuremath{\hookleftarrow}},
    frame=none,
    showtabs=false,
    showspaces=false,
    showstringspaces=false,
    prebreak={},
    keywordstyle=\color[rgb]{0.627,0.126,0.941},
    commentstyle=\color[rgb]{0.133,0.545,0.133},
    stringstyle=\color[rgb]{01,0,0},
    captionpos=t,
    escapeinside={(\%}{\%)}
}

\begin{document}

\begin{center}

    
    \section*{Summax} % ganti judul soal

    \begin{tabular}{ | c c | }
        \hline
        Batas Waktu  & 2s \\    % jangan lupa ganti time limit
        Batas Memori & 512MB \\  % jangan lupa ganti memory limit
        \hline
    \end{tabular}
\end{center}

\subsection*{Deskripsi}

Suatu hari, Gema yang merupakan peserta Gemastik 2024 menemukan sebuah array $A$ dengan panjang $N$. Selain itu, dia juga menemukan bilangan lain $K$. Dia dapat melakukan transformasi pada array tersebut

Dia dapat memilih $K$ atau kurang ( atau bahkan 0 ) elemen dari array tersebut. Apabila ada pasangan elemen terpilih yang bersampingan, maka kedua elemen tersebut dihapus dan digantikan dengan elemen baru yang merupakan penjumlahan dari kedua elemen tersebut. Lalu, untuk semua elemen - elemen yang terpilih tersebut, nilainya akan digantikan menjadi kuadrat dari nilai tersebut. 

Setelah melakukan transformasi tersebut, maka Gema dapat mencari total point yang dia dapatkan dari array tersebut yang merupakan penjumlahan dari elemen - elemen dari array yang ditransformasikan tersebut

Contohnya, ada array $A = [5,10,6,23,98,1,4]$ , $N = 7$ , dan $K = 3$. Gema dapat memilih elemen pada indeks $2,3,$ dan $5$. Array tersebut pertama - tama akan berubah menjadi  $ [5,16,23,98,1,4]$, lalu berubah menjadi  $ [5,256,23,9604,1,4]$. Lalu, poin yang didapatkan adalah 9893

Bantu gema untuk mencari total poin maksimum yang dia bisa dapatkan dari array tersebut

\subsection*{Format Masukan}

Baris pertama terdiri dari dua bilangan bulat positif $N$ ($1 \leq N \leq 10^{5}$) dan $K$ ($0 \leq K \leq 10$), yang masing - masing menyatakan panjang dari array tersebut serta bilangan $K$.

Baris kedua terdiri dari $N$ bilangan bulat $A_{i}$ ($-10^7 \leq A_{i} \leq 10^7$) yang menyatakan elemen - elemen dari array tersebut

\subsection*{Format Keluaran}

Bilangan bulat yang menyatakan total poin maksimum yang Gema bisa dapatkan dari array tersebut

\begin{multicols}{2}
\subsection*{Contoh Masukan 1}
\begin{lstlisting}
5 2
100 2 3 -33 -4000 
\end{lstlisting}
\columnbreak
\subsection*{Contoh Keluaran 1}
\begin{lstlisting}
16265194
\end{lstlisting}
\vfill
\null
\end{multicols}


\subsection*{Penjelasan}

Gema dapat memilih indeks $1$ dan $5$ sehingga array menjadi $[100,2,3,-4033]$ lalu menjadi $[100,2,3,16.265.089]$ sehingga total poin yang didapatkan adalah $16265194$  

\end{document}