\documentclass{article}

\usepackage{geometry}
\usepackage{amsmath}
\usepackage{graphicx, eso-pic}
\usepackage{listings}
\usepackage{hyperref}
\usepackage{multicol}
\usepackage{fancyhdr}
\pagestyle{fancy}
\fancyhf{}
\hypersetup{ colorlinks=true, linkcolor=black, filecolor=magenta, urlcolor=cyan}
\geometry{ a4paper, total={170mm,257mm}, top=10mm, right=20mm, bottom=20mm, left=20mm}
\setlength{\parindent}{0pt}
\setlength{\parskip}{0.3em}
\renewcommand{\headrulewidth}{0pt}

\rfoot{\thepage}
\fancyhf{} % sets both header and footer to nothing
\renewcommand{\headrulewidth}{0pt}
\lfoot{\textbf{CP PRAGEMASTIK ITB XVII}}
\pagenumbering{gobble}

\fancyfoot[CE,CO]{\thepage}
\lstset{
    basicstyle=\ttfamily\small,
    columns=fixed,
    extendedchars=true,
    breaklines=true,
    tabsize=2,
    prebreak=\raisebox{0ex}[0ex][0ex]{\ensuremath{\hookleftarrow}},
    frame=none,
    showtabs=false,
    showspaces=false,
    showstringspaces=false,
    prebreak={},
    keywordstyle=\color[rgb]{0.627,0.126,0.941},
    commentstyle=\color[rgb]{0.133,0.545,0.133},
    stringstyle=\color[rgb]{01,0,0},
    captionpos=t,
    escapeinside={(\%}{\%)}
}

\begin{document}

\begin{center}

    
    \section*{Penyihir Rajin} % ganti judul soal

    \begin{tabular}{ | c c | }
        \hline
        Batas Waktu  & 1s \\    % jangan lupa ganti time limit
        Batas Memori & 256MB \\  % jangan lupa ganti memory limit
        \hline
    \end{tabular}
\end{center}

\subsection*{Deskripsi}

Frieren adalah seorang elf penyihir yang sangat kuat yang menguasai N jenis sihir. Namun, karena musuh-musuhnya juga kuat, dia perlu mempelajari sihir-sihir baru yang bermunculan. Pada setiap bulannya, Frieren dengan rajinnya mempelajari K sihir baru. Selain itu, Frieren yang jenius juga dapat menggabungkan sihir barunya dengan sihir yang telah ia pelajari. Karena menurut Frieren sihir lama sudah tidak relevan, dia akan melupakan sihir yang telah dia kuasai selama 3 bulan.

Bantulah Frieren dalam menghitung jumlah sihir yang ia kuasai dalam T bulan modulo \textbf{998244353}!

\subsection*{Format Masukan}

Baris pertama terdiri dari tiga bilangan bulat positif $N$ ($1 \leq N \leq 10^{9}$), $K$ ($1 \leq K \leq 10^{9}$), dan $T$ ($0 \leq T \leq 10^{18}$) yang masing-masing menyatakan jumlah sihir yang dikuasai Frieren di bulan ke-0, jenis sihir baru yang dipelajari Frieren dalam setiap bulannya, dan bulan dimana jumlah sihir yang dikuasai Frieren ditanyakan pada soal.


\subsection*{Format Keluaran}

Suatu bilangan bulat yang menyatakan banyak sihir yang Frieren kuasai modulo \textbf{998244353}.

\begin{multicols}{2}
\subsection*{Contoh Masukan 1}
\begin{lstlisting}
1 3 4
\end{lstlisting}
\columnbreak
\subsection*{Contoh Keluaran 1}
\begin{lstlisting}
501
\end{lstlisting}
\vfill
\null
\end{multicols}

\begin{multicols}{2}
\subsection*{Contoh Masukan 2}
\begin{lstlisting}
5 5 0
\end{lstlisting}
\columnbreak
\subsection*{Contoh Keluaran 2}
\begin{lstlisting}
5
\end{lstlisting}
\vfill
\null
\end{multicols}

\subsection*{Penjelasan}

Pada contoh 1, Frieren di awal telah menguasai 1 jenis sihir. Lalu untuk setiap bulannya:

\begin{enumerate}
\item Mempelajari $3$ jenis sihir baru, menghasilkan $3  \times 1$ sihir kombinasi. Total $1 + 3 + 3 = 7$.
\item Mempelajari $3$ jenis sihir baru, menghasilkan $3  \times 7$ sihir kombinasi. Total $7 + 3 + 21 = 31$.
\item Mempelajari $3$ jenis sihir baru, menghasilkan $3  \times 31$ sihir kombinasi, dan melupakan $1$ jenis sihir yang dia kuasai sejak bulan ke-0. Total $31 + 3 + 93 - 1 = 126$.
\item Mempelajari $3$ jenis sihir baru, menghasilkan $3  \times 126$ sihir kombinasi, dan melupakan $3 + 3$ jenis sihir yang dia kuasai sejak bulan ke-0. Total $126 + 3 + 378 - 6 = 501$.
\end{enumerate}

Pada contoh 2, Frieren di awal telah menguasai 5 jenis sihir.

\end{document}