\documentclass{article}

\usepackage{geometry}
\usepackage{amsmath}
\usepackage{graphicx, eso-pic}
\usepackage{listings}
\usepackage{hyperref}
\usepackage{multicol}
\usepackage{fancyhdr}
\pagestyle{fancy}
\fancyhf{}
\hypersetup{ colorlinks=true, linkcolor=black, filecolor=magenta, urlcolor=cyan}
\geometry{ a4paper, total={170mm,257mm}, top=10mm, right=20mm, bottom=20mm, left=20mm}
\setlength{\parindent}{0pt}
\setlength{\parskip}{0.3em}
\renewcommand{\headrulewidth}{0pt}

\rfoot{\thepage}
\fancyhf{} % sets both header and footer to nothing
\renewcommand{\headrulewidth}{0pt}
\lfoot{\textbf{CP PRAGEMASTIK ITB XVII}}
\pagenumbering{gobble}

\fancyfoot[CE,CO]{\thepage}
\lstset{
    basicstyle=\ttfamily\small,
    columns=fixed,
    extendedchars=true,
    breaklines=true,
    tabsize=2,
    prebreak=\raisebox{0ex}[0ex][0ex]{\ensuremath{\hookleftarrow}},
    frame=none,
    showtabs=false,
    showspaces=false,
    showstringspaces=false,
    prebreak={},
    keywordstyle=\color[rgb]{0.627,0.126,0.941},
    commentstyle=\color[rgb]{0.133,0.545,0.133},
    stringstyle=\color[rgb]{01,0,0},
    captionpos=t,
    escapeinside={(\%}{\%)}
}

\begin{document}

\begin{center}

    
    \section*{Kocok Kartu} % ganti judul soal

    \begin{tabular}{ | c c | }
        \hline
        Batas Waktu  & 1s \\    % jangan lupa ganti time limit
        Batas Memori & 256MB \\  % jangan lupa ganti memory limit
        \hline
    \end{tabular}
\end{center}

\subsection*{Deskripsi}

Terdapat sebuah tumpukan kartu di meja yang dinomori dari 1 sampai n.



Sebuah shuffle kartu dilakukan dengan membagi kartu menjadi dua bagian yang berukuran sama (bagian atas dan bawah), lalu mulai menaruh kartu dari bagian bawah, kartu dari bagian atas, dan seterusnya secara berseling.



\href{https://en.wikipedia.org/wiki/Shuffling#/media/File:Riffle_shuffle.gif}{Video ilustrasi shuffle}


Cara shuffle (elemen paling kiri adalah kartu paling bawah):

[1, 2, 3, 4, 5, 6, 7, 8] → [5, 1, 6, 2, 7, 3, 8, 4] → [7, 5, 3, 1, 8, 6, 4, 2]

\

Cari bilangan bulat positif terkecil k, sehingga kartu berukuran n di-shuffle sebanyak k kali kembali ke urutan semula.

\subsection*{Format Masukan}

Satu bilangan bulat genap $N$ $(2 \leq N \leq 10^{10})$,

\subsection*{Format Keluaran}

Satu bilangan bulat positif k, jumlah shuffle terkecil sehingga kartu kembali ke urutan semula.

\begin{multicols}{2}
\subsection*{Contoh Masukan 1}
\begin{lstlisting}
2
\end{lstlisting}
\columnbreak
\subsection*{Contoh Keluaran 1}
\begin{lstlisting}
2
\end{lstlisting}
\vfill
\null
\end{multicols}

\begin{multicols}{2}
\subsection*{Contoh Masukan 1}
\begin{lstlisting}
6
\end{lstlisting}
\columnbreak
\subsection*{Contoh Keluaran 1}
\begin{lstlisting}
3
\end{lstlisting}
\vfill
\null
\end{multicols}


\end{document}