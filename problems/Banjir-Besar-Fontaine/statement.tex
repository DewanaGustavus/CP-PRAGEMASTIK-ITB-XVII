\documentclass{article}

\usepackage{geometry}
\usepackage{amsmath}
\usepackage{graphicx, eso-pic}
\usepackage{listings}
\usepackage{hyperref}
\usepackage{multicol}
\usepackage{fancyhdr}
\pagestyle{fancy}
\fancyhf{}
\hypersetup{ colorlinks=true, linkcolor=black, filecolor=magenta, urlcolor=cyan}
\geometry{ a4paper, total={170mm,257mm}, top=10mm, right=20mm, bottom=20mm, left=20mm}
\setlength{\parindent}{0pt}
\setlength{\parskip}{0.3em}
\renewcommand{\headrulewidth}{0pt}

\rfoot{\thepage}
\fancyhf{} % sets both header and footer to nothing
\renewcommand{\headrulewidth}{0pt}
\lfoot{\textbf{CP PRAGEMASTIK ITB XVII}}
\pagenumbering{gobble}

\fancyfoot[CE,CO]{\thepage}
\lstset{
    basicstyle=\ttfamily\small,
    columns=fixed,
    extendedchars=true,
    breaklines=true,
    tabsize=2,
    prebreak=\raisebox{0ex}[0ex][0ex]{\ensuremath{\hookleftarrow}},
    frame=none,
    showtabs=false,
    showspaces=false,
    showstringspaces=false,
    prebreak={},
    keywordstyle=\color[rgb]{0.627,0.126,0.941},
    commentstyle=\color[rgb]{0.133,0.545,0.133},
    stringstyle=\color[rgb]{01,0,0},
    captionpos=t,
    escapeinside={(\%}{\%)}
}

\begin{document}

\begin{center}

    
    \section*{Banjir Besar Fontaine} % ganti judul soal

    \begin{tabular}{ | c c | }
        \hline
        Batas Waktu  & 1s \\    % jangan lupa ganti time limit
        Batas Memori & 256MB \\  % jangan lupa ganti memory limit
        \hline
    \end{tabular}
\end{center}

\subsection*{Deskripsi}

Ramalan cuaca di Fontaine mengatakan akan terjadi hujan lebat selama satu bulan. Untuk itu, Bu Furi selaku presiden Fontaine memerintahkan Menteri Pekerjaan Umum Fontaine, Bu Didot, untuk mencari estimasi kerugian yang didapatkan jika terjadi banjir di suatu daerah.

Fontaine terdiri atas banyak wilayah berbentuk persegi. Diberikan sebuah peta grid $P$ berukuran $N \times M$. $P_{i,j}$ menunjukkan ketinggian wilayah fontaine di petak $(i, j)$. Sebuah petak bersebelahan dengan empat petak yang berbagi sisi dengan petak tersebut. Dengan kata lain, petak $(i, j)$ bersebelahan dengan petak $(i \pm 1, j \pm 1)$.

Banjir dapat berpindah ke tempat yang tidak lebih tinggi dari sumber banjir. Banjir hanya dapat berpindah ke petak yang bersebelahan.

Untuk seluruh petak, bantulah Bu Didot mencari jumlah wilayah yang terdampak jika banjir dimulai dari petak tersebut.

\subsection*{Format Masukan}

Baris pertama berisi dua bilangan bulat $N$ dan $M$ $(1 \leq N \times M \leq 400000)$

$N$ baris berikutnya berisi $M$ bilangan bulat $P_{i,j}$ $(1 \leq P_{i,j} \leq 10^9)$

\subsection*{Format Keluaran}

$N$ baris, masing-masing berisi $M$ bilangan bulat $C_{i,j}$ yang menunjukkan jumlah wilayah terdampak banjir, apabila banjir dimulai dari petak $(i,j)$

\begin{multicols}{2}
\subsection*{Contoh Masukan 1}
\begin{lstlisting}
1 3
5 2 4
\end{lstlisting}
\columnbreak
\subsection*{Contoh Keluaran 1}
\begin{lstlisting}
3 1 2
\end{lstlisting}
\vfill
\null
\end{multicols}


\subsection*{Penjelasan}

Banjir dari petak $(1,3)$ dapat berpindah ke petak $(1,2)$ karena ketinggian sumber banjir $(4)$ $\geq$ ketinggian petak $(2)$, namun tidak dapat berpindah ke petak $(1,1)$ karena ketinggian sumber banjir $<$ ketinggian petak.

\end{document}