\documentclass{article}

\usepackage{geometry}
\usepackage{amsmath}
\usepackage{graphicx, eso-pic}
\usepackage{listings}
\usepackage{hyperref}
\usepackage{multicol}
\usepackage{fancyhdr}
\pagestyle{fancy}
\fancyhf{}
\hypersetup{ colorlinks=true, linkcolor=black, filecolor=magenta, urlcolor=cyan}
\geometry{ a4paper, total={170mm,257mm}, top=10mm, right=20mm, bottom=20mm, left=20mm}
\setlength{\parindent}{0pt}
\setlength{\parskip}{0.3em}
\renewcommand{\headrulewidth}{0pt}

\rfoot{\thepage}
\fancyhf{} % sets both header and footer to nothing
\renewcommand{\headrulewidth}{0pt}
\lfoot{\textbf{CP PRAGEMASTIK ITB XVII}}
\pagenumbering{gobble}

\fancyfoot[CE,CO]{\thepage}
\lstset{
    basicstyle=\ttfamily\small,
    columns=fixed,
    extendedchars=true,
    breaklines=true,
    tabsize=2,
    prebreak=\raisebox{0ex}[0ex][0ex]{\ensuremath{\hookleftarrow}},
    frame=none,
    showtabs=false,
    showspaces=false,
    showstringspaces=false,
    prebreak={},
    keywordstyle=\color[rgb]{0.627,0.126,0.941},
    commentstyle=\color[rgb]{0.133,0.545,0.133},
    stringstyle=\color[rgb]{01,0,0},
    captionpos=t,
    escapeinside={(\%}{\%)}
}

\begin{document}

\begin{center}

    
    \section*{Brankas Aneh} % ganti judul soal

    \begin{tabular}{ | c c | }
        \hline
        Batas Waktu  & 1s \\    % jangan lupa ganti time limit
        Batas Memori & 256MB \\  % jangan lupa ganti memory limit
        \hline
    \end{tabular}
\end{center}

\subsection*{Deskripsi}

Pada suatu hari yang tidak cerah, sekolah menengah atas Abydos menemukan bahwa mereka terlilit hutang besar. Shiroko dan teman-temannya memutuskan untuk membobol suatu bank untuk membayar hutang sekolah mereka.

Di hari pembobolan, Shiroko menemukan suatu brankas yang terlihat berisi barang sangat berharga yang dijaga dengan kunci kombinasi sepanjang $N$ dengan kondisi awal masing-masing angka adalah $X_i$ (kunci dapat bernilai negatif). Teman Shiroko sebelumnya telah menemukan kertas berisi $N$ buah angka $Y_i$ yang diduga besar adalah kunci brankas. Namun, mengubah angka dalam kunci brankas tidak semudah yang mereka kira. Brankas tersebut menyediakan $N$ buah tombol $A_i$ dan $N$ buah tombol $B_i$. Apabila tombol $A_i$ ditekan, nilai $X_{i-1}$ dan $X_{i+1}$ akan bertambah sebesar 1. Sebaliknya, apabila tombol $B_i$ ditekan, nilai $X_{i-1}$ dan $X_{i+1}$ akan berkurang sebesar 1.

Bantulah Shiroko dan teman-temannya menentukan apakah mereka dapat menghasilkan kombinasi $Y$ dari kondisi awal $X$!

\subsection*{Format Masukan}

Baris pertama terdiri dari satu bilangan bulat positif $T$ ($1 \leq N \leq 10^{7}$)  yang banyak kasus.

Setiap kasus terdiri atas 3 baris. Baris pertama terdiri dari satu bilangan bulat positif $N$ ($1 \leq N \leq 10^{7}$)  yang menyatakan panjang kunci kombinasi.
$2$ baris berikutnya berisi nilai $X_i$ ($-10^{5} \leq X_i \leq 10^{5}$) dan $Y_i$ ($-10^{5} \leq Y_i \leq 10^{5}$) yang masing-masing menyatakan kondisi awal brankas dan perkiraan kunci brankas.

\subsection*{Format Keluaran}

Setiap kasus memiliki keluaran satu kata yaitu "Ya" apabila kunci brankas dapat dicapai dari kondisi awal dan "Tidak" apabila sebaliknya.

\subsection*{Batasan Tambahan}
$\sum_{i=1}^T N_i \leq 1 \times 10^{7}$

\begin{multicols}{2}
\subsection*{Contoh Masukan 1}
\begin{lstlisting}
2
5
4 2 1 4 6
6 4 2 6 5
5
1 2 3 4 5
6 6 6 6 6
\end{lstlisting}
\columnbreak
\subsection*{Contoh Keluaran 1}
\begin{lstlisting}
Ya
Tidak
\end{lstlisting}
\vfill
\null
\end{multicols}


\subsection*{Penjelasan}

Pada contoh 1, kunci brankas dapat dicapai dengan melakukan aksi berikut:

\begin{enumerate}
\item Menekan tombol $A_1$ sebanyak 2 kali
\item Menekan tombol $A_2$ sebanyak 2 kali
\item Menekan tombol $B_3$ sebanyak 1 kali
\end{enumerate}

Pada contoh 2, dapat ditunjukkan bahwa kunci brankas tidak akan mungkin dapat dicapai dari kondisi awal.

\end{document}