\documentclass{article}

\usepackage{geometry}
\usepackage{amsmath}
\usepackage{graphicx, eso-pic}
\usepackage{listings}
\usepackage{hyperref}
\usepackage{multicol}
\usepackage{fancyhdr}
\pagestyle{fancy}
\fancyhf{}
\hypersetup{ colorlinks=true, linkcolor=black, filecolor=magenta, urlcolor=cyan}
\geometry{ a4paper, total={170mm,257mm}, top=10mm, right=20mm, bottom=20mm, left=20mm}
\setlength{\parindent}{0pt}
\setlength{\parskip}{0.3em}
\renewcommand{\headrulewidth}{0pt}

\rfoot{\thepage}
\fancyhf{} % sets both header and footer to nothing
\renewcommand{\headrulewidth}{0pt}
\lfoot{\textbf{CP PRAGEMASTIK ITB XVII}}
\pagenumbering{gobble}

\fancyfoot[CE,CO]{\thepage}
\lstset{
    basicstyle=\ttfamily\small,
    columns=fixed,
    extendedchars=true,
    breaklines=true,
    tabsize=2,
    prebreak=\raisebox{0ex}[0ex][0ex]{\ensuremath{\hookleftarrow}},
    frame=none,
    showtabs=false,
    showspaces=false,
    showstringspaces=false,
    prebreak={},
    keywordstyle=\color[rgb]{0.627,0.126,0.941},
    commentstyle=\color[rgb]{0.133,0.545,0.133},
    stringstyle=\color[rgb]{01,0,0},
    captionpos=t,
    escapeinside={(\%}{\%)}
}

\begin{document}

\begin{center}

    
    \section*{String Keapus} % ganti judul soal

    \begin{tabular}{ | c c | }
        \hline
        Batas Waktu  & 2s \\    % jangan lupa ganti time limit
        Batas Memori & 512MB \\  % jangan lupa ganti memory limit
        \hline
    \end{tabular}
\end{center}

\subsection*{Deskripsi}

Diberi string s dan t, serta diketahui string t dikonstruksi dari string s berturut turut yang isinya hilang di beberapa tempat (lihat contoh masukan).

Tentukan apakah t valid (s dapat mengkonstruksi t dengan aturan di atas) dan jika valid, berapa jumlah s minimum yang mungkin untuk membentuk t.
\

\subsection*{Format Masukan}
Baris pertama berisi string s yang berisi \textit{lowercase english letter} dengan panjang N $(1 \leq N \leq 10^{5})$

Baris kedua berisi string t yang berisi \textit{lowercase english letter} dengan panjang M $(1 \leq M \leq 10^{5})$


\subsection*{Format Keluaran}

Satu bilangan bulat positif k, yang bernilai jumlah s minimum yang mungkin untuk membentuk t jika string t valid, dan bernilai -1 jika string t tidak valid.

\begin{multicols}{2}
\subsection*{Contoh Masukan 1}
\begin{lstlisting}
gema
gemagemagggma
\end{lstlisting}
\columnbreak
\subsection*{Contoh Keluaran 1}
\begin{lstlisting}
5
\end{lstlisting}
\vfill
\null
\end{multicols}

\begin{multicols}{2}
\subsection*{Contoh Masukan 1}
\begin{lstlisting}
abcdef
bcdefg
\end{lstlisting}
\columnbreak
\subsection*{Contoh Keluaran 1}
\begin{lstlisting}
-1
\end{lstlisting}
\vfill
\null
\end{multicols}

\subsection*{Penjelasan}

Pada contoh masukan 1, string t dapat dikonstruksi dengan mengambil huruf dari string s sebagai berikut:
GeMA GEMA Gema Gema GeMA

Pada contoh masukan 2, string t tidak dapat dikonstruksi dari string s.
\
\end{document}